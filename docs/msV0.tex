\documentclass{emulateapj}
%\usepackage[utf8]{inputenc}

\title{Oblation of Kepler planets}
\author{}
\date{September 2013}

\begin{document}

\maketitle

\section{Introduction}

TBD.

\section{Transit of An Oblate Planet}

In this paper, $f_\perp$ and $\theta_\perp$ are used to stand for the projected flattening and obliquity of a planet respectively, the same as in \citet{Carter2010}. Fig.~\ref{oblate-signals} shows the signals of oblateness for different $f_\perp$, $\theta_\perp$ and orbital inclination $i$, which are based on the algorithm developed in \citet{Carter2010}.

\begin{figure}
\centering
\includegraphics[widht=0.8\textwidth]{oblate-signals.eps}
% the figure hasn't been finished.
\caption{} \label{oblate-signals}
\end{figure}

In this paper, we focus on the oblateness due to spin deformation (why, and the requirement on orbital period)

\section{Detectability in \textit{Kepler} Data}

Impose an oblateness of 0.1 to the planet of kplr-6603043; 
fit the mock data and recover the signal.

\section{Fitting and Constraining the Sample Planets}


\begin{thebibliography}{}
\bibitem[Carter \& Winn(2010)]{Carter2010} Carter, J.~A., \& Winn, J.~N.\ 2010, \apj, 709, 1219

\end{thebibliography}

\end{document}

