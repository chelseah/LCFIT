\documentclass{emulateapj}
%\usepackage[utf8]{inputenc}

\title{Oblation of Kepler planets}
\author{}
\date{September 2013}

\begin{document}

\maketitle

\section{Introduction}

TBD.

\section{Model}
\subsection{Transit signals of oblate planets}

In this paper, $f_\perp$ and $\theta_\perp$ are used to stand for the projected flattening and obliquity of a planet respectively, the same as in \citet{Carter2010}. Fig.~\ref{oblate-signals} shows the signals of oblateness for different $f_\perp$, $\theta_\perp$ and orbital inclination $i$, which are based on the algorithm developed in \citet{Carter2010}.

\begin{figure}
\centering
\includegraphics[widht=0.8\textwidth]{oblate-signals.eps}
% the figure hasn't been finished.
\caption{} \label{oblate-signals}
\end{figure}

In this paper, we focus on the oblateness due to spin deformation (why, and the requirement on orbital period)



\subsection{MCMC technique}
\subsection{Validation of the dection capability}


Impose an oblateness of 0.1 to the planet of kplr-6603043; 
fit the mock data and recover the signal.

\section{Data}

We select Kepler planet candidates from Q1-Q16 with the following criterion: 

a) planet radius smaller than 2.0 $R_J$; 

b) planet period longer than 15 days; %might need to reduce to 7 days.

c) the transit is not grazing ($b<0.8$?);  

d) expected normalized signal to noise higher than?

We define the normalized signal to noise to be 
\begin{equation}
NSN = (\frac{r_p/r_{\rm star}}{0.1})^2(\frac{ootv}{200{\rm ppm}})^{-1}\sqrt{N_{\rm tran}},
\end{equation}
in which, the out of transit variation $ootv$ is computed with the averaged 
standard deviation before transit and after transit in a window with size of 
half the transit duration. We use the theoretical residual amplitude 200 ppm 
of a jupiter like planet transiting a solar type star with 
$theta_\prep=45^{\circ}$ in our estimation. This residual amplitude scale with 
the sqare of the planet star radius ratio.  
The actual detection signal to noise should be $\sqrt{N_{\rm p}}*NSN$, in 
which the number of points contribute to the residual signal per transit. 
We assume most of the residual signal comes from ingress and egress of the 
transit. 
Therefore, for a jupiter size planet transit the midplane of a solar type 
star with a period of 100 day, we expect the normalized signal to noize to be 
7.4, and the actualy detection signal to noise to be 74.8.  


We detrend all the available public Kepler \lcs\ for our analysis of the above targets. 
To remove the stellar variability, we use the raw flux $(\rm SAP\_FLUX)$ obtained from the MAST archive
\footnote{http://archive.stsci.edu/kepler/data$\_$search/search.php}, with the 
out-of-transit variations corrected by the following steps from 
\citet{Huang:2013}:

a) removal of bad data points;

b) correction of systematics due to various phenomena of the space craft, such as safe modes and tweaks;

c) a set of cosine functions with minimum period of 1 day; 

d) a 7th order polynomial fit over the out-of-transit regions. 


\section{Result}

\section{Discussion}


\begin{thebibliography}{}
\bibitem[Carter \& Winn(2010)]{Carter2010} Carter, J.~A., \& Winn, J.~N.\ 2010, \apj, 709, 1219
\bibitem[{{Huang} {et~al.}(2013){Huang}, {Bakos}, \& {Hartman}}]{Huang:2013}
{Huang}, X., {Bakos}, G.~{\'A}., \& {Hartman}, J.~D. 2013, \mnras, 429, 2001

\end{thebibliography}

\end{document}

